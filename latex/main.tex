\documentclass[a4paper]{article}
\usepackage{package}
\title{Report laboratorio}
\begin{document}
\maketitle

\section{27/02/2024}
taglio chip e pulizia in acetone e successivamente in isopropanolo. Asciugatura con aria compressa.
\section{28/02/2024}
Spin coating: 4000 rpm per 1 min e cottura 70 ° C per 3 min. 20\% pmma in etil lattato e 4\% elettrolita per quantità di polimero.
Caratterizzazione dei chip prodotti al microscopio ottico, le immagini acquisiti si possono trovare a \url{www.github.com/mfiaschi5/tesitriennale/foto/chipspinned/1}
\begin{table}[h!]
    \centering
    \begin{tabular}{c c c}
    \hline
        Codice & Elettrolita & Commenti \\
        \hline
         A & \ce{C8(MIM)2(PF6)2} & La soluzione di elettrolita risultava opaca\\
          B & \ce{C8(MIM)BF4} & \\
           C & \ce{C6(MIM)2(Br)2}& \\
            D & \ce{C6(MIM)TFSI}& \\
            E & \ce{C8(MIM)PF6}& Il polimero dopo la cottura dava segni di disomogenità\\
             F & \ce{C3(MIM)2(Br)2}& \\
    \end{tabular}
    \caption{El}
    \label{tab:my_label}
\end{table}

Controllo del sistema di misura e del sistema per il vuoto
\end{document}